\documentclass[a4paper,10pt, leqno]{article}
\usepackage[utf8]{inputenc}
\usepackage{mathtools}
\usepackage{amsfonts}
\usepackage{amssymb}
\usepackage{amsthm}
\usepackage{graphicx}
\usepackage{indentfirst}
\usepackage{array}
\usepackage{bm}
\usepackage{caption}
\usepackage{algorithm}
\usepackage[noend]{algpseudocode}
\usepackage{bbm}
%\usepackage{algorithmic}

\newcommand{\restr}[1]{|_{#1}}

\theoremstyle{definition}
\def\blankpage{%
      \null%
      \clearpage}

\let\oldref\ref
\renewcommand{\ref}[1]{(\oldref{#1})}

%opening
\title{MAC0325 Combinatorial Optimization \\
        \large Assignment 3 }
\author{Pedro Gigeck Freire \\
        10737136}
\date{December 14, 2020}

%\setlength{\parindent}{0.5em}
%\setlength{\parskip}{0.1em}
\begin{document}

\maketitle

\section*{Exercise 1}

Let $u : A \to \mathbb{R} \cup \{+\infty\}$ be a capacity function given by $u(a) \coloneqq +\infty$ for each $a \in A$, and let $
\beta \coloneqq 1$.

We claim that the Shortest Walk Problem (SWP) on $(D, c, r, s)$ is homomorphically equivalent to the Min-Cost Flow Problem (MCFP) on $(D, c, u, r, s, \beta)$

First, let us set some notation.

Let $X \coloneqq \{ W : W \text{ is a } rs\text{-walk in } D \}$ be the feasible set of SWP on $(D, c, r, s)$ and let $Y \coloneqq \{ f \in R_+^A : f \leq u \text{ and } B_Df = \beta(e_s - e_r)\}$ be the feasible set of MCFP on $(D, c, u, r, s, \beta)$.
 
\newtheorem{proposition}{Proposition}
\begin{proposition}
    Let $\varphi : X \to Y$ be a function given by 
    \begin{equation*}
        \tag{1.1} \varphi(W) \coloneqq \mathbbm{1}_W \text{, for each } W \in X.
    \end{equation*}
    
Then, SWP $\xrightarrow{\varphi}$ MCFP. 
\end{proposition}
\begin{proof}
Let $W \in X$. We have.
\begin{align*}
 c^{\intercal}\varphi(W) &= c^{\intercal}\mathbbm{1}_W &\text{by (1.1)} \\
 &= c(W) &\text{by (11.13) from the lectures}\\
\end{align*}

Indeed, the walk $W$ and the flow obtained $\varphi(W)$ have the same objective value. Hence $\varphi$ is a homomorphism.
\end{proof}


\begin{proposition}
 Let $f \in Y$ be a $rs$-flow in $D$. Then there is a walk $W \in X$ such that $c^{\intercal}f \geq c(W)$.  
\end{proposition}

\begin{proof}

If $G$ has any negative cycle $C \coloneqq \langle v_0, a_1, v_1, ..., a_l, v_l \rangle$, then there is a walk $R$ from $r$ to $v_0$ and a walk $S$ from $v_0$ to $s$, because $r \rightsquigarrow v \rightsquigarrow s$ for each $v \in V$.

So one can take a $rs$-walk $W$ with cost arbitrarily low, by taking
$$W \coloneqq R \cdotp (\prod_{i \in [k]}{C}) \cdotp S $$
for some $k \in \mathbb{N}$ arbitrarily high, so that $c^{\intercal}f \geq c(W)$.

Now suppose $G$ has no negative cycles.

 By the exercise 18.10 from the lectures (Decomposition of Flows), there are: a set $\mathcal{C}$ of cycles of $G$, a set $\mathcal{P}$ of $rs$-paths of $G$, a vector $y \in R_+^{\mathcal{C}}$ and a vector $x \in R_+^{\mathcal{P}}$ such that
 
\begin{align*}
  \tag{1.2} f = \sum_{C \in \mathcal{C}}{y_C \mathbbm{1}_C} + \sum_{P \in \mathcal{P}}{x_P \mathbbm{1}_P}, \\
  \tag{1.3} \mathbbm{1}^{\intercal}x = value(f) = \beta = 1
\end{align*}

 
 Now, let $W \in \mathcal{P}$ attain a path of minimun cost, i.e.,

\begin{align*}
  c(W) &= \min_{P \in \mathcal{P}}{c(P)} \implies \\
  \tag{1.4} c(W) &\geq c(P) \text{ for each } P \in \mathcal{P}
\end{align*}

Then

\begin{align*}
  c^{\intercal}f &= c^{\intercal}(\sum_{C \in \mathcal{C}}{y_C \mathbbm{1}_C} + \sum_{P \in \mathcal{P}}{x_P \mathbbm{1}_P}) &\text{ by (1.2)}\\
  &= c^{\intercal}\sum_{C \in \mathcal{C}}{y_C \mathbbm{1}_C} + c^{\intercal}\sum_{P \in \mathcal{P}}{x_P \mathbbm{1}_P} \\
  &= \sum_{C \in \mathcal{C}}{y_C (c^{\intercal} \mathbbm{1}_C)} + \sum_{P \in \mathcal{P}}{x_P (c^{\intercal}\mathbbm{1}_P)} \\
  &\geq \sum_{C \in \mathcal{C}}{y_C} + \sum_{P \in \mathcal{P}}{x_P (c^{\intercal}\mathbbm{1}_P)} &\text{since $G$ has no negative cycles}\\
  &\geq \sum_{P \in \mathcal{P}}{x_P (c^{\intercal}\mathbbm{1}_P)} &\text{since }y \geq 0\\
  &= \sum_{P \in \mathcal{P}}{x_P \sum_{a \in A}{c(a)\mathbbm{1}_P(a)}} \\
  &= \sum_{P \in \mathcal{P}}{x_P \sum_{a \in A}{c(a)[a \in A(P)]}} \\
  &= \sum_{P \in \mathcal{P}}{x_P c(P)} \\
  &\geq \sum_{P \in \mathcal{P}}{x_P c(W)} &\text{ by (1.4)} \\
  &=  c(W)\sum_{P \in \mathcal{P}}{x_P} \\
  &=  c(W)\mathbbm{1}^{\intercal}x \\
  &=  c(W) &\text{ by (1.3)}
\end{align*}
\end{proof}

Note that Proposition 2 builds an implicit homomorphism from MCFP on $(D, c, u, r, s, \beta)$ to SWP on $(D, c, r, s)$. 

Thus, since Proposition 1 builds a homomorphism from SWP to MCFP, these two problems are homomorphically equivalent.


\section*{Exercise 3}

We will build a new graph by adding a new arc from $s$ to $r$ with cost $0$ and lower and upper capacity equal to $\beta$.

Let $a'$ be a new arc (not already in $A$), let $A' \coloneqq A \cup \{a'\}$ and let $\varphi' : A' \to V \times V$ be an extension of $\varphi$ by setting $\varphi'(a') \coloneqq (s, r)$.

Let $D' \coloneqq (V, A', \varphi')$. Let $c' \in \mathbb{R}^{A'}$ be an extension of $c$ by setting 
\begin{equation*}
    \tag{3.1} c'(a') \coloneqq 0 
\end{equation*}

Let $u' : A' \to \mathbb{R}_+ \cup \{+\infty\}$ be an extension of $u$ by setting 

\begin{equation}
    \tag{3.2} u'(a') \coloneqq \beta
\end{equation}

And let $l' \in R_+^{A'}$ be defined by 
$$l'(a) \coloneqq
\left\{
	\begin{array}{ll}
		0  & \mbox{if } a \in A \\
		\beta & \mbox{if } a = a' \\
	\end{array}
\right.$$

Now, we may build our isomorphism.

Let $X \coloneqq \{ f \in R_+^A : f \leq u \text{ and } B_Df = \beta(e_s - e_r)\}$ be the set of feasible $rs$-flows in $D$ (with respect to $u$).

Let $Y \coloneqq \{ f' \in R_+^{A'} : l' \leq f' \leq u'\}$ be the set of feasible circulations in $D'$ (with respect to $l'$ and $u'$).

\paragraph{3.1 The Homomorphism} Let $\psi : X \to Y$ be a function that maps a flow in $D$ to its correspondent circulation in $D'$, i.e.
\begin{align*}
    \tag{3.3} \psi(f) \coloneqq f'
\end{align*}
with
\begin{align*}
    \tag{3.4}
    f'(a) \coloneqq 
    \left\{
        \begin{array}{ll}
            f(a)  & \mbox{if } a \in A \\
            \beta & \mbox{if } a = a' \\
        \end{array}
    \right. \text{ for each } a \in A'
\end{align*}

Note that $\psi$ is well defined, since $f'$ is indeed a feasible circulation with respect to $l'$ and $u'$ by the definition of these vectors. For each $a \in A$,  we have 
$$l'(a) = 0 \leq f(a) = f'(a) = f(a) \leq u(a) = u'(a)$$
and for $a = a'$ we have 
$$l'(a') = f'(a') = u'(a') = \beta.$$

\newtheorem{lemma}{Lemma}
\begin{lemma}
 $\psi$, as defined in 3.1, is a homomorphism from the Min-Cost Flow Problem on $(D, c, u, r, s, \beta)$ to the Min-Cost Circulation Problem on $(D', c', l', u')$.
\end{lemma}
\begin{proof}
\begin{align*}
 {c'}^{\intercal}\psi(f) &= {c'}^{\intercal}f' &\text{ by (3.3)}\\
 &= \sum_{a \in A'}{c'(a)f'(a)} \\
 &= \sum_{a \in A}{c'(a)f'(a)} + c'(a') &\text{ by definition of } A'\\
 &= \sum_{a \in A}{c'(a)f'(a)} &\text{ by (3.1)}\\
 &= \sum_{a \in A}{c(a)f(a)} &\text{ since $c'$ and $f'$ are extensions}\\
 &= c^{\intercal}f
\end{align*}

Thus, $\psi(f)$ and $f$ have the same objective value, so $\psi$ is a homomorphism.
\end{proof}

\begin{proposition}
 $\psi$, as defined in 3.1, is injective.
\end{proposition}
\begin{proof}
 Let $f, g \in X$ such that $\psi(f) = \psi(g)$. We have
 \begin{align*}
  \psi(f) = \psi(g)  &\implies f' = g' &\text{by (3.3)} \\
  &\implies f'(a) = g'(a) \text{ for each } a \in A' \\
  &\implies f'(a) = g'(a) \text{ for each } a \in A &\text{ because } A \subset A' \\
  &\implies f(a) = g(a) \text{ for each } a \in A &\text{ by (3.4)} \\
  &\implies f = g
 \end{align*}

 Hence $\psi$ is injective.
\end{proof}

\begin{proposition}
 $\psi$, as defined in 3.1, is surjective.
\end{proposition}
\begin{proof}
 Let $f' \in Y$ and set $f \coloneqq f'\restr{A}$. We claim that $f \in X$.
 
First, we have that $f$ is a $rs$-flow in $D$ that respects 
 $$0 = l'(a) \leq f(a) \leq u'(a) = u(a) \text{ for each } a \in A$$ 
 
 Now, note that $f'(a') = \beta$, because
 $$
 \beta = l'(a) \leq f'(a) \leq u'(a) = \beta
 $$
 
Then, $B_Df = \beta(e_s - e_r)$, since $B_Df' = 0$, so that when we remove the arc $a'$ the only affected vertices are $s$ and $r$, so the excess of flow in these vertices is $f'(a') = \beta$ in $s$ and $-\beta$ in $r$.

These facts shows that $f \in X$.

 Thus, it is straighfoward from the definition of $\psi$ and the fact that $f'(a) = \beta$ that $f' = \psi(f)$

 Hence, each $f' \in Y$ is the image of some $f \in X$, so $\psi$ is surjective.
\end{proof}

\begin{lemma}
 $\psi$, as defined in 3.1, is a bijection.
\end{lemma}
\begin{proof}
 Immeadiate from Propositions 3 and 4.
\end{proof}

\begin{lemma}
Let $\psi$ be defined as in 3.1. Then the inverse function $\psi^{-1} : Y \to X $ is a homomorphism from the Min-Cost Circulation Problem on $(D', c', l', u')$ to the Min-Cost Flow Problem on $(D, c, u, r, s, \beta)$.
\end{lemma}
\begin{proof}
 As we showed in the proof of Proposition 4, for each $f' \in Y$ we have 
 \begin{equation*}
 \tag{3.5} \psi^{-1}(f') = f'\restr{A}.
 \end{equation*}

 
 So, for each circulation $f' \in Y$
 \begin{align*}
  c^\intercal \psi^{-1}(f') &= c^\intercal f'\restr{A} &\text{ by (3.5)} \\
  &= \sum_{a \in A}{c(a)f'(a)} \\
  &= \sum_{a \in A}{c'(a)f'(a)} &\text{ by definition of } c' \\
  &= \sum_{a \in A}{c'(a)f'(a)} + c'(a')f'(a) &\text{ by (3.1)} \\
  &= \sum_{a \in A'}{c'(a)f'(a)} &\text{ by definition of } A' \\
  &= {c'}^\intercal f'
 \end{align*}

Hence $f'$ and $\psi^{-1}(f')$ is a homomorphism. 
\end{proof}

\newtheorem{theorem}{Theorem}
\begin{theorem}
 $\psi$, as defined in 3.1, is a isomorphism.
\end{theorem}
\begin{proof}
 Immediate from Lemmas 1, 2 and 3.
\end{proof}


Furthermore, note that if $u$ and $\beta$ are integral, then $u'$ and $l'$ are both integral, and if the flow $f \in X$ is integral, than the circulation $\psi(f)$ is also integral by the definition of $\psi$.

\blankpage
\section*{Exercise 14}

We start with $M_0 = \emptyset$ and $y_0 = 0$

\paragraph{$\mathbf{t = 0}$}
(Matching update)

$P_0 = \langle 3, 9 \rangle$ 

$M_1 = \{\{ 3, 9 \}\}$ 

\paragraph{$\mathbf{t = 1}$}

(Dual update)

$K_1 = \{ 3 \}$

$d_1 = (1, 1, 0, 1, 1, 0, 0, 0, 0, 0)$

$\lambda_1 = 1$

$y_2 = (1, 1, 0, 1, 1, 0, 0, 0, 0, 0)$


\paragraph{$\mathbf{t = 2}$}

(Dual update)

$K_2 = \{ 9 \}$

$d_2 = (1, 1, 1, 1, 1, 0, 0, 0, -1, 0)$

$\lambda_2 = 1$

$y_3 = (2, 2, 1, 2, 2, 0, 0, 0, -1, 0)$

\paragraph{$\mathbf{t = 3}$}

(Matching update)

$P_3 = \langle 1, 9, 3, 10 \rangle$ 

$M_4 = \{\{ 1, 9 \}, \{3, 10\}\}$ 

\paragraph{$\mathbf{t = 4}$}

(Dual update)

$K_4 = \{ 1, 3 \}$

$d_4 = (0, 1, 0, 1, 1, 0, 0, 0, 0, 0)$

$\lambda_4 = 2$

$y_5 = (2, 4, 1, 4, 4, 0, 0, 0, -1, 0)$

\paragraph{$\mathbf{t = 5}$}

(Dual update)

$K_5 = \{ 3, 9 \}$

$d_5 = (1, 1, 0, 1, 1, 0, 0, 0, -1, 0)$

$\lambda_5 = 1$

$y_6 = (3, 5, 1, 5, 5, 0, 0, 0, -2, 0)$

\paragraph{$\mathbf{t = 6}$}

(Dual update)

$K_6 = \{ 9, 10 \}$

$d_6 = (1, 1, 1, 1, 1, 0, 0, 0, -1, -1)$

$\lambda_6 = 1$

$y_7 = (4, 6, 2, 6, 6, 0, 0, 0, -3, -1)$

\paragraph{$\mathbf{t = 7}$}

(Matching update)

$P_7 = \langle 2, 10, 3, 6 \rangle$ 

$M_8 = \{\{ 2, 10 \}, \{3, 6\}, \{ 1, 9 \}\}$ 

\paragraph{$\mathbf{t = 8}$}

(Dual update)

$K_8 = \{ 9, 10, 3 \}$

$d_8 = (1, 1, 0, 1, 1, 0, 0, 0, -1, -1)$

$\lambda_8 = 2$

$y_9 = (6, 8, 2, 8, 8, 0, 0, 0, -5, -3)$

\paragraph{$\mathbf{t = 9}$}

(Dual update)

$K_9 = \{ 6, 9, 10 \}$

$d_9 = (1, 1, 1, 1, 1, -1, 0, 0, -1, -1)$

$\lambda_9 = 3$

$y_{`10} = (9, 11, 5, 11, 11, -3, 0, 0, -8, -6)$

\paragraph{$\mathbf{t = 10}$}

(Matching update)

$P_{10} = \langle 5, 9, 1, 6, 3, 8 \rangle$ 

$M_{11} = \{\{2, 10\}, \{ 5, 9 \}, \{1, 6 \}, \{ 3, 8 \}\}$ 

\paragraph{$\mathbf{t = 11}$}

(Dual update)

$K_{11} = \{ 1, 2, 3, 5 \}$

$d_{11} = (0, 0, 0, 1, 0, 0, 0, 0, 0, 0)$

$\lambda_{11} = 4$

$y_{12} = (9, 11, 5, 15, 11, -3, 0, 0, -8, -6)$

\paragraph{$\mathbf{t = 12}$}

(Dual update)

$K_{12} = \{ 1, 3, 5, 10 \}$

$d_{12} = (0, 1, 0, 1, 0, 0, 0, 0, 0, -1)$

$\lambda_{12} = 1$

$y_{13} = (9, 12, 5, 16, 11, -3, 0, 0, -8, -7)$

\paragraph{$\mathbf{t = 13}$}

(Dual update)

$K_{13} = \{ 3, 6, 9, 10 \}$

$d_{13} = (1, 1, 0, 1, 1, -1, 0, 0, -1, -1)$

$\lambda_{13} = 1$

$y_{14} = (10, 13, 5, 17, 12, -4, 0, 0, -9, -8)$

\paragraph{$\mathbf{t = 14}$}

(Dual update)

$K_{14} = \{ 6, 8, 9, 10 \}$

$d_{14} = (1, 1, 1, 1, 1, -1, 0, -1, -1, -1)$

$\lambda_{14} = 3$

$y_{15} = (13, 16, 8, 20, 15, -7, 0, -3, -12, -11)$

\paragraph{$\mathbf{t = 15}$}

(Matching update)

$P_{15} = \langle 4, 10, 2, 6, 1, 8, 3, 7 \rangle$ 

$M_{16} = \{\{4, 10\}, \{2, 6\}, \{1, 8\}, \{3, 7\}, \{5, 9\}\} \\$


Thus, the optimal value found was 39 and the solutions found were 

$$
M = \{\{4, 10\}, \{2, 6\}, \{1, 8\}, \{3, 7\}, \{5, 9\}\}
$$
$$
y = (13, 16, 8, 20, 15, -7, 0, -3, -12, -11)
$$

\end{document}
