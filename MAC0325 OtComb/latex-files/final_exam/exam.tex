\documentclass[a4paper,10pt, leqno]{article}
\usepackage[utf8]{inputenc}
\usepackage{mathtools}
\usepackage{amsfonts}
\usepackage{amssymb}
\usepackage{amsthm}
\usepackage{graphicx}
\usepackage{indentfirst}
\usepackage{array}
\usepackage{bm}
\usepackage{caption}
\usepackage{algorithm}
\usepackage[noend]{algpseudocode}
\usepackage{bbm}
%\usepackage{algorithmic}

\newcommand{\restr}[1]{|_{#1}}

\theoremstyle{definition}
\def\blankpage{%
      \null%
      \clearpage}

\let\oldref\ref
\renewcommand{\ref}[1]{(\oldref{#1})}


%opening
\title{MAC0325 Combinatorial Optimization \\
        \large Final Exam}
\author{Pedro Gigeck Freire \\
        10737136}
\date{December 16, 2020}

%\setlength{\parindent}{0.5em}
%\setlength{\parskip}{0.1em}
\begin{document}

\maketitle

\section*{Exercise 1}


\paragraph{Necessity:}

Let $f \in \mathbb{R}^A$ be a vector such that $0 \leq f \leq u$ and excess$_f \leq b$.
Then, for each $S \in V$ we have
\begin{align*}
 b(S) &\geq \text{excess}_f(S) \\
 &= B_Df(S) &\text{ by (12.5) from lectures} \\
 &= f(\delta^{\text{in}}(S)) - f(\delta^{\text{out}}(S)) &\text{ by (12.2) from lectures}\\
 &\geq f(\delta^{\text{in}}(S)) - u(\delta^{\text{out}}(S)) &\text{ since } f \leq u
\end{align*}

The inequality above, together with the fact that $f \geq 0$, shows
\begin{align*}
 b(S) + u(\delta^{\text{out}}(S)) \geq f(\delta^{\text{in}}(S)) &\geq 0
\end{align*}

And this completes the proof for necessity.

\paragraph{Sufficiency:} Our hypothesis is that 
\begin{align*}
 b(S) + u(\delta^{\text{out}}(S)) \geq 0 \text{ for each } S \subseteq V
\end{align*}

\paragraph{\textbf{1.1 The Auxiliar Digraph.}}
Let $V_- \coloneqq \{ v \in V : b'(v) < 0\}$ be the set of producers and $V_+ \coloneqq \{ v \in V : b'(v) \geq 0\}$ be the set of consumers. Note that $V = V_- \cup V_+$ and it is a disjoint union.

Let $r$ be a new vertex (not already in $V$) and set $V' \coloneqq V \cup \{ r \}$. Let $A' \coloneqq V_- \sqcup A \sqcup V_+$ and let $\varphi' : A' \to V' \times V'$ be defined by

$$
\varphi'((i, x)) \coloneqq
\left\{
	\begin{array}{ll}
		(r, x)  & \mbox{if } i = 1 \text{ (whence } x \in V_-)\\
		\varphi(x)  & \mbox{if } i = 2 \text{ (whence } x \in A) \\
		(x, r)  & \mbox{if } i = 3 \text{ (whence } x \in V_+)
	\end{array}
\right.
$$

Let $l' \in \mathbb{R}^{A'}$ be defined by
$$
l'((i, x)) \coloneqq
\left\{
	\begin{array}{ll}
		-b(x)  & \mbox{if } i = 1 \text{ (whence } x \in V_-) \\
		0  & \mbox{if } i = 2 \text{ (whence } x \in A) \\
		b(x)  & \mbox{if } i = 3 \text{ (whence } x \in V_+)
	\end{array}
\right.
$$

Let $u' : A' \to \mathbb{R} \cup \{+\infty\}$ be defined by
$$
u'((i, x)) \coloneqq
\left\{
	\begin{array}{ll}
		-b(x)  & \mbox{if } i = 1 \text{ (whence } x \in V_-) \\
		u(x)  & \mbox{if } i = 2 \text{ (whence } x \in A) \\
		b(x)  & \mbox{if } i = 3 \text{ (whence } x \in V_+)
	\end{array}
\right.
$$

Now, build the digraph $D' \coloneqq (V', A', \varphi')$. 

\paragraph{\textbf{1.2 The Cuts in D'.}}
Let us analyse what happened to the sets $\delta^{\text{in}}$ and $\delta^{\text{out}}$ of each subset of $V$ when we moved from $D$ to $D'$

For each $S \subseteq V$, the set $\delta_{D'}^{\text{in}}(S)$ includes the (corresponding) arcs that were in the original digraph, and we added arcs going inside the producers (from $r$). 

The set $\delta_{D'}^{\text{out}}(S)$ includes the (corresponding) arcs that were in the original digraph, and we just added arcs going outside the consumers (to $r$). 

Then, set $S_- \coloneqq S \cap V_-$ and $S_+ \coloneqq S \cap V_+$ (note that $S = S_- \cup S_+$ and it is a disjoint union), we have
\begin{align*}
\tag{1.1}
    \delta_{D'}^{\text{in}}(S) &= (\{2\}\times\delta_{D}^{\text{in}}(S)) \cup (\{1\} \times S_-) \\
\tag{1.2}
    \delta_{D'}^{\text{out}}(S) &= (\{2\}\times\delta_{D}^{\text{out}}(S)) \cup (\{3\} \times S_+)
\end{align*}

Next, we treat the case $S' \subseteq V'$ with $r \in S'$. Set $S \coloneqq S' \setminus V'$.

In this case, the arcs going inside of $S$ are the (corresponding) ones that were going inside of $S \setminus \{r\}$ in $D$ together with the new arcs that go inside $r$, that comes from the consumers in $V_+$ that are not in $S_+$, i.e.,
\begin{align*}
\tag{1.3}
    \delta_{D'}^{\text{in}}(S') &= (\{2\}\times\delta_{D}^{\text{in}}(S)) \cup (\{3\} \times (V_+ \setminus S_+)) 
\end{align*}

And the arcs going outside of $S$ are the (corresponding) ones that were going out of $S \setminus \{r\}$ in $D$ together with the new arcs going out of $r$ to the producers in $V_-$ that are not in $S_-$, i.e.,
\begin{align*}
\tag{1.4}
    \delta_{D'}^{\text{out}}(S') &= (\{2\}\times\delta_{D}^{\text{out}}(S)) \cup (\{1\} \times (V_- \setminus S_-)) 
\end{align*}

\paragraph{1.3 Lower and Upper bounds in D'.}
For each $S' \subseteq V'$, if $r \notin V'$ (whence $S ' \subseteq V$), then
\begin{align*}
 l'(\delta_{D'}^{\text{in}}(S')) &= l'((\{2\}\times\delta_{D}^{\text{in}}(S')) \cup (\{1\} \times S_{-}')) &\text{by (1.1)} \\
 &= l'(\{2\}\times\delta_{D}^{\text{in}}(S')) + l'(\{1\} \times S_{-}') \\
 &= 0 + (-b(S_-')) &\text{ by definition of }l'
\end{align*}

Analogously,
\begin{align*}
 u'(\delta_{D'}^{\text{out}}(S')) &= u'((\{2\}\times\delta_{D}^{\text{out}}(S')) \cup (\{3\} \times S_+')) &\text{by (1.2)}\\
 &= u'(\{2\}\times\delta_{D}^{\text{out}}(S')) + u'(\{3\} \times S_+') \\
 &= u(\delta_{D}^{\text{out}}(S')) + b(S_+') &\text{ by definition of }u' 
\end{align*}

Otherwise, if $r \in S'$, set $S \coloneqq S' \setminus \{r\}$  
\begin{align*}
 l'(\delta_{D'}^{\text{in}}(S')) &= l'((\{2\}\times\delta_{D}^{\text{in}}(S)) \cup (\{3\} \times (V_+ \setminus S_+)) &\text{by (1.3)} \\
 &= l'(\{2\}\times\delta_{D}^{\text{in}}(S)) + l'(\{3\} \times V_+) - l'(\{3\} \times S_+)\\
 &= 0 + b(V_+) - b(S_+) &\text{ by definition of }l'
\end{align*}

\begin{align*}
 u'(\delta_{D'}^{\text{out}}(S')) &= u'((\{2\}\times\delta_{D}^{\text{out}}(S)) \cup (\{1\} \times (V_- \setminus S_-)) &\text{by (1.4)}\\
 &= u'(\{2\}\times\delta_{D}^{\text{out}}(S)) + u'(\{1\} \times V_-) - u'(\{1\} \times S_-) \\
 &= u(\delta_{D}^{\text{out}}(S)) - b(V_-) + b(S_-) &\text{ by definition of }u' 
\end{align*}

Summing up, for each $S' \subseteq V'$, if $r \notin S$, then
\begin{align*}
 \tag{1.5} l'(\delta_{D'}^{\text{in}}(S')) &= -b(S_-') \\
 \tag{1.6} u'(\delta_{D'}^{\text{out}}(S')) &= u(\delta_{D}^{\text{out}}(S')) + b(S_+')
\end{align*}
and if $r \in S'$, set $S \coloneqq S' \setminus \{r\} $
\begin{align*}
 \tag{1.7} l'(\delta_{D'}^{\text{in}}(S')) &= b(V_+) - b(S_+) \\
 \tag{1.8} u'(\delta_{D'}^{\text{out}}(S')) &= u(\delta_{D}^{\text{out}}(S)) - b(V_-) + b(S_-)
\end{align*}


\newtheorem{proposition}{Proposition}
\begin{proposition}
 Restore the context from $1.1$. Then there is a feasible circulation $f'$ in $D'$ with respect to $l'$ and $u'$.
\end{proposition}
\begin{proof}
For each $S' \subseteq V'$, if $r \notin S'$
\begin{align*}
 u'(\delta_{D'}^{\text{out}}(S')) - l'(\delta_{D'}^{\text{in}}(S')) &= u(\delta_{D}^{\text{out}}(S')) + b(S_+') - (-b(S_-')) &\text{ by (1.5) and (1.6)}\\
 &= u(\delta_D^{\text{out}}(S')) + b(S') \\
 &\geq 0 &\text{ by hypothesis} 
\end{align*}

If $r \in S$, set $S \coloneqq S' \setminus \{r\}$, then
\begin{align*}
 u'(\delta_{D'}^{\text{out}}(S')) - l'(\delta_{D'}^{\text{in}}(S')) &= u(\delta_{D}^{\text{out}}(S)) - b(V_-) + b(S_-) - b(V_+) + b(S_+) &\text{ by (1.7) and (1.8)}\\
 &= u(\delta_D^{\text{out}}(S)) - b(V) + b(S) \\
 &\geq -b(V) &\text{ by hypothesis} \\
 &= 0
\end{align*}

This two cases shows that 
$$ u'(\delta_{D'}^{\text{out}}(S')) \geq l'(\delta_{D'}^{\text{in}}(S')) \text{ for each } S' \subseteq V'$$

Thus, we can apply Hoffman's Theorem, verifying that (17.24)(ii) fails, so there is a feasible circulation $f'$ in $D'$ with respect to $l'$ and $u'$.
\end{proof}

\paragraph{\textbf{1.3 Circulation to Transshipment}}

Let $f'$ be a feasible circulation $D'$ with respect to $l'$ and $u'$ (we know it exists by Proposition 1).
Let $f : A \to \mathbb{R}$ be defined by $f(a) \coloneqq f'((2, a))$ for each $a \in A$. 

We claim that $f$ is a $b$-transshipment in $D$.

First, it is simple to verify that $0 \leq f \leq u$, because for each $a \in A$ $$0 = l'((2, a)) \leq f'((2, a)) = f(a) = f'((2, a)) \leq u'((2, a)) = u(a)$$

Next, we need to show that excess$_f = b$. For each $v \in V$, if $v \in V_-$ then (1.1) shows that 
\begin{align*}
\tag{1.5}
\delta_{D'}^{\text{in}}(v) &= (\{2\}\times\delta_{D}^{\text{in}}(v)) \cup \{(1, v)\} \\
\tag{1.6}
\delta_{D'}^{\text{out}}(v) &= \{2\}\times\delta_{D}^{\text{out}}(v)
\end{align*}

Then
\begin{align*}
 \text{excess}_f(v) &= B_Df(v) \\
&= f(\delta_D^{\text{in}}(v)) - f(\delta_D^{\text{out}}(v)) \\
&= f'(\{2\} \times \delta_D^{\text{in}}(v)) - f'(\{2\} \times \delta_D^{\text{out}}(v)) &\text{ by definition of }f\\
&= f'(\delta_{D'}^{\text{in}}(v) \setminus \{(1, v)\}) - f'(\delta_{D'}^{\text{out}}(v)) &\text{ by (1.5) and (1.6) } \\
&= f'(\delta_{D'}^{\text{in}}(v)) - f'((1, v) - f'(\delta_{D'}^{\text{out}}(v)) \\
&= B_{D'}f'(v) - f'((1, v)) \\
&= - f'((1, v)) &\text{ since $f'$ is a circulation} \\
&= b(v) &\text{by definition of } f'
\end{align*}

Using the same logic, if $v \in V_+$, then (1.2) shows 
\begin{align*}
\tag{1.7}
\delta_{D'}^{\text{in}}(v) &= (\{2\}\times\delta_{D}^{\text{in}}(v)) \\
\tag{1.8}
\delta_{D'}^{\text{out}}(v) &= \{2\}\times\delta_{D}^{\text{out}}(v) \cup \{(3, v)\} 
\end{align*}

Then
\begin{align*}
 \text{excess}_f(v) &= B_Df(v) \\
&= f(\delta_D^{\text{in}}(v)) - f(\delta_D^{\text{out}}(v)) \\
&= f'(\{2\} \times \delta_D^{\text{in}}(v)) - f'(\{2\} \times \delta_D^{\text{out}}(v)) &\text{ by definition of }f\\
&= f'(\delta_{D'}^{\text{in}}(v)) - f'(\delta_{D'}^{\text{out}}(v)  \setminus \{(3, v)\}) &\text{ by (1.7) and (1.8) } \\
&= f'(\delta_{D'}^{\text{in}}(v)) - f'(\delta_{D'}^{\text{out}}(v)) + f'((3, v)  \\
&= B_{D'}f'(v) + f'((3, v)) \\
&= f'((3, v)) &\text{ since $f'$ is a circulation} \\
&= b(v) &\text{by definition of } f'
\end{align*}

So, excess$_f(v) = b(v)$ for each $v \in V$, thus $f$ is a $b$-transshipment in $D$. It concludes the proof for sufficiency.

\blankpage
\section*{Exercise 2}

We will use the degree contraint idea, similar to section 13.2 from the lecture notes.

\paragraph{\textbf{2.1 The Auxiliar Digraph D'.}}

Let $r, s$ be new vertices (not in $V \sqcup A$) and let 
$$V' \coloneqq \{r, s\} \cup (A \sqcup V)$$
$$A' \coloneqq A \sqcup A \sqcup A \sqcup V$$

Let $\varphi' : A' \to V' \times V'$ be an incidence function defined by
$$
\varphi'((i, x)) \coloneqq
\left\{
	\begin{array}{ll}
		(r, (1, x))  & \mbox{if } i = 1 \text{ (whence } x \in A)\\
		((1, x), (2, v)) & \mbox{if } i = 2 \text{ (whence } x \in A) \text{ and } \varphi(x) = (v, w) \\
		((1, x), (2, w))  & \mbox{if } i = 3 \text{ (whence } x \in A) \text{ and } \varphi(x) = (v, w) \\
		((2, x), s)  & \mbox{if } i = 4 \text{ (whence } x \in V)
	\end{array}
\right.
$$

Let $D' \coloneqq (V', A', \varphi') $ be a digraph.

Let $c : A' \to \mathbb{R}$ be a cost function defined by
$$
c((i, x)) \coloneqq
\left\{
	\begin{array}{ll}
		k(x)  & \mbox{if } i = 3 \text{ (whence } x \in A \text{ and } \\
		0  & \mbox{if } i \neq 3
	\end{array}
\right.
$$

Define deg $: V \to \mathbb{N}$ by deg$(v) \coloneqq |\delta_D^\text{in}(v)| + |\delta_D^\text{out}(v)|$ (degree of $v$ in the underlying graph of $D$).

Let $u : A' \to \mathbb{R} \cup \{+\infty\}$ be a capacity function defined by
$$
u((i, x)) \coloneqq
\left\{
	\begin{array}{ll}
		1  & \mbox{if } i = 1, 2, 3 \text{ (whence } x \in A) \\
		\text{deg}(x)/2 & \mbox{if } i = 4 \text{ (whence } x \in V) \\
	\end{array}
\right.
$$

\paragraph{\textbf{2.2 The Cuts in D'.}}


Let us discover what are the sets $\delta_{D'}^\text{in}(v)$ and $\delta_{D'}^\text{out}(v)$ for each $v \in V'$. The results below come direct from the definition of $\varphi'$. Looking at figure 17.2 from the lectures may be helpful.

For each $v' \in V'$

If $v' = r$, then there is one arc from $r$ to each vertex of the form $(1, a)$, i.e.,
\begin{align*}
 \tag{2.1}
 \begin{split}
 \delta_{D'}^\text{in}(r) &= \emptyset \\
 \delta_{D'}^\text{out}(r) &= \{1\} \times A
 \end{split}
\end{align*}

If $v' = s$, then there is one arc from each vertex of the form $(2, v)$ to $s$, i.e.,
\begin{align*}
 \tag{2.2}
 \begin{split}
 \delta_{D'}^\text{in}(s) &= \{4\} \times V \\
 \delta_{D'}^\text{out}(s) &= \emptyset
 \end{split}
\end{align*}

If $v' \in \{1\} \times A$, let $(1, a) \coloneqq v'$, then there is one arc entering $v'$ (from $r$) and two arcs leaving $v'$, one to $(2, v)$ and other to $(2, w)$ with $(v, w) \coloneqq \varphi(a)$, then
\begin{align*}
 \tag{2.3}
 \begin{split}
 \delta_{D'}^\text{in}(v) &= \{(1, a)\} \\
 \delta_{D'}^\text{out}(v) &= \{(2, a), (3, a)\}
 \end{split}
\end{align*}

And if $v' \in \{2\} \times V$, let $(2, v) \coloneqq v$, then there is one arc leaving $v'$ (to $s$) and deg$(v)$ arcs entering at $v'$ (some corresponding to  $\delta_{D}^\text{in}(v)$ and the others corresponding $\delta_{D}^\text{out}(v)$), i.e.,
\begin{align*}
 \tag{2.4}
 \begin{split}
 \delta_{D'}^\text{in}(v) &= (\{2\} \times \delta_{D}^\text{in}(v)) \cup (\{3\} \times \delta_{D}^\text{out}(v)\} \\
 \delta_{D'}^\text{out}(v) &=  \{(4, v)\}
 \end{split}
\end{align*}

\begin{proposition}
 Let $B \subseteq A$. If $D_B$ has a closed Eulerian trail, then, for each $v \in V$
 $$|\delta_D^\text{in}(v) \setminus B| + |\delta_D^\text{out}(v) \cap B| = \text{deg}(v)/2.$$
\end{proposition}
\begin{proof}
 By the way that $D_B$ is defined, we have that the arcs entering some vertex $v \in V$ are the ones that enter $v$ in $D$ that were not inverted (are not in $B$) together with the arcs that leave $v$ in $D$ that were inverted (are in $B$). So
 \begin{align*}
  \tag{2.5} \delta_{D_B}^\text{in}(v) = (\delta_D^\text{in}(v) \setminus B) \cup (\delta_D^\text{out}(v) \cap B)
 \end{align*}
 
Now, since $D_B$ does not changed the ends of any arc (only its order), $$|\delta_{D_B}^\text{in}(v)| + |\delta_{D_B}^\text{out}(v)| = |\delta_{D}^\text{in}(v)| + |\delta_{D}^\text{out}(v)| = \text{deg}(v)$$.

And since $D_B$ has an eulerian trail, then $|\delta_{D_B}^\text{in}(v)| = |\delta_{D_B}^\text{out}(v)|$ for each $v \in V$, so 
\begin{align*}
 |\delta_{D_B}^\text{in}(v)| + |\delta_{D_B}^\text{out}(v)| = \text{deg}(v) &\implies \\
 2|\delta_{D_B}^\text{in}(v)| =\text{deg}(v) &\implies \\
 |\delta_D^\text{in}(v) \setminus B| + |\delta_D^\text{out}(v) \cap B| = \text{deg}(v)/2 
\end{align*}
\end{proof}

\paragraph{2.3 Building a Flow.}
Let $B \subseteq A$. Let $f_B \in \mathbb{R}_+^{A'}$ be defined by
$$
f_B((i, x)) \coloneqq
\left\{
	\begin{array}{ll}
		1 & \mbox{if } i = 1 \text{ (whence } x \in A)\\
		\lbrack x \notin B \rbrack & \mbox{if } i = 2 \text{ (whence } x \in A)\\
		\lbrack x \in B \rbrack & \mbox{if } i = 3 \text{ (whence } x \in A)\\
		\text{deg}(x)/2 & \mbox{if } i = 4 \text{ (whence } x \in V)
	\end{array}
\right.
$$

Informally, the idea is that for each arc $a \in A$, let $(v, w) \coloneqq \varphi(a)$, if $a \notin B$, then there is flow in the arc $(2, a)$ but not in the arc $(3, a)$. And if $a \in B$, then there is flow in $(3, a)$ but not in $(2, a)$.

Note that for each $a \in A$
\begin{align*}
\tag{2.6}
 f_B((2, a)) + f_B((3, a)) = \lbrack a \notin B \rbrack + \lbrack a \in B \rbrack = 1
\end{align*}

\newtheorem{theorem}{Theorem}
\begin{theorem}{\textbf{(From B to Flow)}}
 Restore the context from 2.1 and let $B \subseteq A$. If $D_B$ has a closed Eulerian trail then there is a feasible $rs$-flow $f \in R_+^{A'}$ in $D'$ with respect to $u$. Furthermore value$(f) = |A|$ and $c^\intercal f = k(B)$.
\end{theorem}
\begin{proof}
Set $f \coloneqq f_B$ as in 2.3. 

We claim that $f$ is the desired flow.

First, it is straighfoward from the definition of $f$ and $u$ that $f \leq u$.

Next, we will show that $B_{D'}f = |A|(e_s - e_r)$

For each $v' \in V'$, if $v' = r$, then 
\begin{align*}
B_{D'}f(r) &= f(\delta_{D'}^\text{in}(r)) - f(\delta_{D'}^\text{out}(r)) &\text{by (12.2) from lectures} \\
&= f(\emptyset) - f(\{1\} \times A) &\text{by (2.1)}\\
&= -|A| &\text{by definition of }f
\end{align*}

If $v' = s$, then 
\begin{align*}
B_{D'}f(s) &= f(\delta_{D'}^\text{in}(s)) - f(\delta_{D'}^\text{out}(s)) &\text{by (12.2) from lectures} \\
&= f(\{4\} \times V) - f(\emptyset) &\text{by (2.2)}\\
&= |A| &\text{by definition of }f
\end{align*}

If $v' \in \{1\} \times A$, let $(1, a) \coloneqq v'$, then 
\begin{align*}
B_{D'}f(v') &= f(\delta_{D'}^\text{in}(v')) - f(\delta_{D'}^\text{out}(v')) &\text{by (12.2) from lectures} \\
&= f(\{(1, a)\}) - f(\{(2, a), (3, a)\}) &\text{by (2.3)}\\
&= 1 - 1 &\text{by definition of }f \text{ and (2.6)} \\
&= 0
\end{align*}

Last, if $v' \in \{2\} \times V$, let $(2, v) \coloneqq v'$ then 
\begin{align*}
B_{D'}f(v') &= f(\delta_{D'}^\text{in}(v')) - f(\delta_{D'}^\text{out}(v')) \\
&= f(\{2\} \times \delta_D^\text{in}(v)) + f(\{3\} \times \delta_D^\text{out}(v)) -  f((4, v))&\text{by (2.4)} \\
&= \sum_{a \in \delta_D^\text{in}(v)}{f((2, a))} + \sum_{a \in \delta_D^\text{out}(v)}{f((3, a))} -  f((4, v)) \\
&= \sum_{a \in \delta_D^\text{in}(v)}{\lbrack a \notin B \rbrack} + \sum_{a \in \delta_D^\text{out}(v)}{\lbrack a \in B \rbrack} - \text{deg}(v)/2 &\text{by definition of }f\\ 
&= |\delta_D^\text{in}(v) \setminus B| + |\delta_D^\text{out}(v) \cap B| - \text{deg}(v)/2 \\
&= 0 &\text{by proposition 2}
\end{align*}

Indeed, the 4 cases above shows that $B_{D'}f = |A|(e_s - e_r)$.

Now, it remains to prove that $c^\intercal f = k(B)$.

It is straighfoward from the definition of $c$ that 
\begin{align*}
 \tag{2.7} c((i, x)) \neq 0 \iff i = 3, x \in A
\end{align*}

Finally, we have
\begin{align*}
 c^\intercal f &= \sum_{a \in A'}{c(a)f(a)} \\
 &= \sum_{x \in A}{c((3, x))f((3, x))} &\text{by (2.7)} \\
  &= \sum_{x \in A}{k(x)\lbrack x \in B \rbrack }  &\text{ by definition of $c$ and $f$} \\
  &= k(B)
\end{align*}
\end{proof}

\begin{theorem}{\textbf{(From Flow to B)}}
Restore the context from 2.1 and let $f \in R_+^{A'}$ be a minimum-cost integral $rs$-flow in $D'$ (with respect to $u$ and cost function $c$) with value$(f) = |A|$.

Let $B \coloneqq \{a \in A : f((3, a)) > 0\}$.

Then $B$ is feasible for the original optimization problem and $k(B) = c^\intercal f$.
\end{theorem}
\begin{proof}
We are assuming that the underlying graph of $D$ is connected, so that in order to prove that $D_B$ has a closed Eulerian trail, it suffices to show that $|\delta_{D_B}^\text{in}(v)| = |\delta_{D_B}^\text{out}(v)|$ for each $v \in V$.

Our strategy will be analyse the flow through each layer of vertices in $D'$.

Starting the first layer from $r$, note that the set $\delta_{D'}^\text{out}(r)$ has capacity
\begin{align*}
 u(\delta_{D'}^\text{out}(r)) &= u(\{1\} \times A)) &\text{by (2.1)} \\
 &= \sum_{a \in A}{u((1, a))} \\
 &= |A| &\text{by definition of }u
\end{align*}

So that each arc in $\delta_{D'}^\text{out}(r)$ is filled to capacity, because value$(f) = |A|$, i.e.,
\begin{align*}
 \tag{2.8}
 f((1, a)) = u((1, a)) = 1 \text{ for each } (1, a) \in \delta_{D'}^\text{out}(r)
\end{align*}

Now, in the next layer, we have that for vertex $v' \in V'$ of the form $v' \coloneqq (1, a)$
\begin{align*}
\tag{2.9}
 f(\delta_{D'}^\text{in}(v)) &= f((1, a)) &\text{by (2.3)}\\
 &= 1 &\text{ by(2.8)}
\end{align*}

And, for each $a \in A$, set $(v, w) \coloneqq \varphi(a)$
\begin{align*}
 f((2, a)) + f((3, a)) &= f(\delta_{D'}^\text{out}((1, a))) &\text{by (2.3)}\\
 &= f(\delta_{D'}^\text{in}((1, a))) &\text{ by flow conservation} \\
 &= 1 &\text{ by (2.9)}
\end{align*}

And since $f$ is integral and $f \leq u$, we have that $f((2, a))$ and $f((3, a))$ are either $0$ or $1$. So the equation above and the definition of $B$ implies that 
\begin{align*}
\tag{2.10}
\begin{split}
 f((2, a)) &= 1 - f((3, a)) \\
 &= \lbrack f((3, a)) = 0 \rbrack \\
 &= \lbrack a \notin B \rbrack
\end{split}
\end{align*}
and
\begin{align*}
\tag{2.11}
\begin{split}
 f((3, a)) &= \lbrack f((3, a)) = 1 \rbrack \\
 &= \lbrack a \in B \rbrack
\end{split}
\end{align*}

Now, in the third layer, for each vertex $v' \in V'$ of the form $v' \coloneqq (2, v)$, 
\begin{align*}
 f(\delta_{D'}^\text{in}((2, v))) &= f((\{2\} \times \delta_{D}^\text{in}(v)) \cup \{3\} \delta_{D}^\text{out}(v))  &\text{by (2.4)}\\
 &= \sum_{a \in \delta_{D}^\text{in}(v)}{f((2, a))} + \sum_{a \in \delta_{D}^\text{out}(v)}{f((3, a))}  \\
 &= \sum_{a \in \delta_{D}^\text{in}(v)}{\lbrack a \notin B \rbrack} + \sum_{a \in \delta_{D}^\text{out}(v)}{\lbrack a \in B \rbrack}  &\text{by (2.10) and (2.11)} \\
 &= |\delta_{D}^\text{in}(v) \setminus B| + |\delta_{D}^\text{out}(v) \cap B| \\
 &= |\delta_{D_B}^\text{in}(v)| &\text{by (2.5)} 
\end{align*}
And 
\begin{align*}
 f(\delta_{D'}^\text{out}((2, v))) &= f((4, v)) &\text{by (2.4)}
\end{align*}

So, the two results above shows that for each $v \in V$
\begin{align*}
 \tag{2.12}
 |\delta_{D_B}^\text{in}(v)| &= f(\delta_{D'}^\text{in}((2, v))) \\
 &= f(\delta_{D'}^\text{out}((2, v))) &\text{ by flow conservation} \\
 &= f((4, v))
\end{align*}

In the last layer, note that the cut formed by $V' \setminus \{s\}$ in $D'$ has capacity 
\begin{align*}
u(\delta_{D'}^{\text{out}}(V' \setminus \{s\})) &= u(\delta_{D'}^{\text{in}}(s)) \\
&= u(\{4\} \times V) &\text{by (2.2)}\\
&= \sum_{v \in V}{u((4, v))} \\
&= \sum_{v \in V}{\text{deg}(v)/2} &\text{by definition of }u \\
&= \frac{1}{2}\sum_{v \in V}{(|\delta_D^\text{in}(v)| + |\delta_D^\text{out}(v)|)} \\
&= \frac{1}{2}(\sum_{v \in V}{|\delta_D^\text{in}(v)|} + \sum_{v \in V}{|\delta_D^\text{out}(v)|)} \\
&= \frac{1}{2}(|A| + |A|) \\
&= |A|
\end{align*}

Then, since value$(f) = |A|$, every arc in $\delta_{D'}^{\text{out}}(V' \setminus \{s\})$ is filled to capacity.
And hence $\delta_{D'}^{\text{out}}(V' \setminus \{s\}) = \delta_{D'}^{\text{in}}(s) = \{4\} \times V$ (by (2.2)) 
\begin{align*}
\tag{2.13}
 f((4, v)) = u((4, v)) = \text{deg}(v)/2 &\text{ (by definition of }u)
\end{align*}

Note that, since the underlying graph of $D$ and $D_B$ are the same, we have that $|\delta_{D_B}^\text{in}(v)| + |\delta_{D_B}^\text{out}(v)| =|\delta_{D}^\text{in}(v)| + |\delta_{D}^\text{out}(v)| $

Finally,
\begin{align*}
|\delta_{D_B}^\text{in}(v)| &= f((4, v)) &\text{by (2.12)} \\
&= \text{deg}(v)/2 &\text{by (12.13)} \\
&= \frac{|\delta_{D}^\text{in}(v)| + |\delta_{D}^\text{out}(v)|}{2} \\
&= \frac{|\delta_{D_B}^\text{in}(v)| + |\delta_{D_B}^\text{out}(v)|}{2}
\end{align*}

And this implies that $|\delta_{D_B}^\text{in}(v)| = |\delta_{D_B}^\text{out}(v)|$. So $D_B$ has a closed Eulerian trail and $B$ is feasible for the original problem.

The fact that $k(B) = c^\intercal f$ comes directly from the definition of B and (2.6).

\end{proof}


\paragraph{\textbf{2.4 Finally, The Algorithm.}} 

The idea is to build $D'$, run min-cost flow algorithm to obtain a flow $f$, and then obtain the set $B$ as in Theorem 2.  

Let MIN-COST-INTEGRAL-FLOW be an algorithm to solve the Minimum-cost Integral Flow Problem and suppose it is efficient. 

Algorithm 1 is our solution to this exercise.

\begin{algorithm}{}
\begin{algorithmic}[1]
\caption{Min-cost Inversions necessary to Eulerian Trail using Flow}
\Procedure{}{$D,k$}:
\State \Comment Define $D'$, $c$ and $u$ as in 2.1
\State $V' \gets \{r, s\} \cup (V \sqcup A)$
\State $A' \gets A \sqcup A \sqcup A\sqcup V$
\State Define $c, u$ and $\varphi'$ as in 2.1
\State $D' \gets (V', A', \varphi')$
\State $f \gets $MIN-COST-INTEGRAL-FLOW$(D', c, u, r, s, |A|)$
\If{$f = \perp$ (Infeasible)}
\State \textbf{return} Infeasible
\Else
\State $B \gets \{a \in A : f((3, a)) > 0\}$
\State \textbf{return} $B$
\EndIf
\EndProcedure
\end{algorithmic}
\end{algorithm}

Since all definitions are made in linear time, MIN-COST-INTEGRAL-FLOW is assumed to be efficient and there is no loops or recursion in Algorithm 1, then it is efficient.

For the correctness, we have that the return statement from line 9 is correct, because if there was a feasible set $B$, then there would be a feasible flow $f$ by Theorem 1.

Furthermore, the set $B$ returned in line 12 is feasible by Theorem 2 and it is optimal, because if it was not, there would be a better set $B'$ with $k(B') < k(B)$, but then the flow $f$ would not be minumum, because Theorem 1 shows that there would be a flow $f'$ with cost $c^\intercal f' = k(B') < k(B) = c^\intercal f$.

\blankpage

\section*{Exercise 3}

The idea is to add arcs from the consumers to the producers, and then find a circulation in this new digraph.

Let $A_{+-} \coloneqq \{(v_+, v_-) : v_+ \in V_+, v_- \in V_-\}$ be the set of pairs of consumers and producers  
Let $A' \coloneqq A \sqcup A_{+-}$.

Define $\varphi' : V \to A'$ as
$$
\varphi'((i, x)) \coloneqq
\left\{
	\begin{array}{ll}
		\varphi(x)  & \mbox{if } i = 1 \text{ (whence } x \in A) \\
		x  & \mbox{if } i = 2 \text{ (whence } x \in A_{+-}
    \end{array}
\right.
$$

Now let $D' \coloneqq (V, A', \varphi')$ be a digraph.

Let $f'$ be a circulation in $D'$.

By Theorem 18.7 from the lecture notes (decomposition of circulations), there is a set $\mathcal{C}$ of cycles in $D'$ and a vector $y \in \mathbb{R}_+^\mathcal{C}$ such that
$$
f'  = \sum_{C \in \mathcal{C}}{y_C\mathbbm{1}_C}
$$

Next, find all cycles such that there is an arc of $A_{+-}$ and remove this arc.

For each $C \in \mathcal{C}$, if there is an arc $a \in A_{+-} \cap A(C)$, set $(v_+, v_-) \coloneqq a$.
Unpack $C$ as $C \coloneqq \langle v_-, a_0, v_0, ..., v_+, a, v_- \rangle$.

Now, we remove $a$ form $C \in $ and get a path $P(C) \coloneqq \langle v_-, a_0, v_0, ..., v_+ \rangle$. So that the arcs of $P(C)$ are not in $A_{+-}$ (since we removed the ones that were in $A_+-$).  Note that $P(C)$ starts in some vertex of $V_-$ and ends in some vertex of $V_+$. 


Let $\mathcal{C}' \coloneqq \{C \in \mathcal{C} : A(C) \cap A_{+-} \neq \emptyset \}$ be the set of cycles that goes through some arc of $A_{+-}$. Then

$$
f = \sum_{C \in (\mathcal{C} \setminus \mathcal{C}') }{x_C \mathbbm{1}_C} + \sum_{C \in \mathcal{C}'}{y_C \mathbbm{1}_{P(C)}}
$$

Organizing the sets, let $\mathcal{P} \coloneqq \{P(C) : C \in \mathcal{C}' \}$ and let $x \in \mathbb{R}_+^\mathcal{P}$ be defined by $x_{P(C)} \coloneqq y_C$. Thus we have

$$
f = \sum_{C \in (\mathcal{C} \setminus \mathcal{C}') }{x_C \mathbbm{1}_C} + \sum_{P \in \mathcal{P}}{x_P \mathbbm{1}_{P}}
$$

So that the sets $\mathcal{C} \setminus \mathcal{C'}$ and $\mathcal{P}$ satisfy the conditions stated.  

\blankpage

\section*{Exercise 4}
We claim that every vertex of $G$ is not isolated.

Since $G$ is loopless and $E \neq \emptyset$, there is one edge $e \in E$ joining two different vertices, implying that $|V| \geq 2$.

Therefore, for each $v \in V$, there is $w \in V$ with $w \neq v$, so that $v$ is in the vertex set of $G - w$. But $G - w$ has a perfect matching, meaning that $v$ is saturated by some edge $e \in E$. Thus 
\begin{align*}
 \delta(v) \neq \emptyset &\text{ for each } v \in V.
\end{align*} 

Now, suppose for the sake of contradiction that $G$ is bipartite and let $U, W \subseteq V$ be such that $G$ is $(U, W)$-bipartite. And since $G$ has no isolated vertices, we have that $V = U \cup W$.

Let $u \in U$ and let $M$ be a perfect matching in $G - u$. Then
\begin{align*}
 \tag{4.1}
 |M| = \frac{|V \setminus \{u\}|}{2} = \frac{|V| - 1}{2} = \frac{|U| + |W| - 1}{2}
 \end{align*} 


Moreover, each vertex $w$ in $W$ is saturated by exactly one edge of $M$ and each edge $e \in M$ saturates one vertex of $W$, because $G - u$ is $(U-W)$-bipartite. Then
\begin{align*}
 \tag{4.2}
   |M| &= |W|
\end{align*}

Now, (4.1) and (4.2) shows that 
\begin{align*}
|W| &= \frac{|U| + |W| - 1}{2}
\end{align*}

Then
\begin{align*}
\tag{4.3}
|W| &= |U| - 1
\end{align*}

Analogously, let $w \in W$ and $M'$ be a perfect matching of $G - w$. Then
\begin{align*}
 \tag{4.4}
 |M'| = \frac{|V \setminus \{w\}|}{2} = \frac{|V| - 1}{2} = \frac{|U| + |W| - 1}{2}
 \end{align*} 

Moreover, each vertex $u$ in $U$ is saturated by exactly one edge of $M'$ and each edge $e \in M'$ saturates one vertex of $U$, because $G - w$ is $(U-W)$-bipartite. Then
\begin{align*}
 \tag{4.5}
   |M'| &= |U|
\end{align*}

Now, (4.4) and (4.5) shows that 
\begin{align*}
|U| &= \frac{|U| + |W| - 1}{2}
\end{align*}

Then
\begin{align*}
\tag{4.6}
|U| &= |W| - 1
\end{align*}

Finally, (4.3) and (4.6) shows that 
\begin{align*}
|W| &= (|W| - 1) - 1 &\implies \\
|W| &= |W| - 2 &\implies \\
0 &= -2
\end{align*}

A contradiction.

Thus, $G$ is not bipartite.


\end{document}
